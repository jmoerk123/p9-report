Today keyword spotting (KWS) is implemented in various systems like voice assistants. For these systems to achieve satisfying performance the models typically rely on a large amount of labeled data. This is limiting these application to situations where only this kind of data is available. To accommodate for this problem [Holgers article] has been looking into using self supervised learning (SSL). 

Self supervised learning (SSL) have existed for a while, and while it has been the same idea driving the SSL in different fields, the implementations has differed [source: Data2vec article, abstract]. Therefor [Data2vec article] has introduced, what they call data2vec, which is a \textit{framework that uses the same learning method for either speech, NLP, or computer vision}.

Since most systems using KWS are implemented places with some or more background noise. If you use speech commands on your phone the background noise could be anything from a bus stopping to people speaking.

In today's society, it is almost a given that new technology have some level of voice assistants, whether it is simple voice commands or integrated voice assistants, like Apple's Siri or Google's Assistant, keyword spotting is a part of it. For the voice assistants, there is an activating keyword. Using a keyword, it is possible to avoid running a more computational expensive automatic speech recognition (ASR) when it is not needed. Keyword spotting (KWS) also has many other applications, such as speech data mining, audio indexing, and phone call routing [KWS overview]. 