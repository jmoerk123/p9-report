% Base packages ----------------------------------------------------------------
\usepackage{ragged2e}
\usepackage{verbatim}
\usepackage[T1]{fontenc}     % adds support for printing accented characters
\usepackage[utf8]{inputenc}  % adds support for inputting accented characters


\usepackage[square]{natbib}  %citation style
\bibliographystyle{apalike} 
\setcitestyle{aysep={:},yysep={;}}
%\let\cite=\citeauthor

\usepackage{hyperref}        % insert clickable hyperlinks
\usepackage{lastpage}        % reference last page (used to count numbered pages)


% Math packages ----------------------------------------------------------------
\usepackage{amsmath,amssymb,stmaryrd,mathtools,mathrsfs,bbm}  

% Other symbols 
\usepackage{textcomp} 
\usepackage{physics}
\usepackage{siunitx}    % si-units
\usepackage{gensymb}

% Figure packages --------------------------------------------------------------
\usepackage[dvipsnames, table, xcdraw]{xcolor}    % font colouring functionality
\usepackage{pgfplots}
\usepackage{graphicx}              % handling of images
\usepackage{booktabs}              % improved tables
\usepackage{array}                 % setting rowheight

\usepackage{algorithm}             % algorithm floats
\usepackage[noend]{algpseudocode}  % pseudocode listings
\usepackage{listings}              % source code listings
\usepackage{enumerate}
\usepackage{enumitem}
\usepackage{colortbl}              % table color
\usepackage{multirow}
\usepackage{makecell}
% Styling packages -------------------------------------------------------------

\usepackage{microtype}             % improve full justification and font kerning
\usepackage{emptypage}             % suppress page numbering on empty pages
\usepackage{fancyhdr}              % easy customisation of headers and footers
\usepackage{titlesec}              % easy customisation of section titles
\usepackage[margin=3cm]{geometry}  % adjust page margins
\usepackage[toc,page]{appendix}    % improved appendix control
\usepackage{framed}                % environments package
\usepackage[framed,amsmath,thmmarks]{ntheorem} % environments package
\usepackage{stfloats}              % linking tables to bottom of page
\usepackage{etoolbox}              % align removal spacing
\usepackage[parfill]{parskip}      % removes indent on new line

% Font packages ----------------------------------------------------------------
\usepackage{lmodern}  % improved default font
%todo
\usepackage[
%  disable,                  % turn off todonotes
  colorinlistoftodos,        % enable a coloured square in the list of todos
  textwidth=\marginparwidth, % set the width of the todonotes
  textsize=scriptsize,       % size of the text in the todonotes
  ]{todonotes}

% Theorem environments -----------------------------------------------------------

% \theoremheaderfont{\normalfont\bfseries}
% \theorembodyfont{\normalfont}
% \theoremstyle{break}
% \def\theoremframecommand{{\color{ForestGreen!50}\vrule width 5pt \hspace{5pt}}}
% \newshadedtheorem{lemma}{Lemma}[chapter]

% \def\theoremframecommand{{\color{red!50}\vrule width 5pt \hspace{5pt}}}
% \newshadedtheorem{example}{Example}[chapter]

% \def\theoremframecommand{{\color{ForestGreen!80}\vrule width 5pt \hspace{5pt}}}
% \newshadedtheorem{corollary}{Corollary}[chapter]

% \def\theoremframecommand{{\color{MidnightBlue!100}\vrule width 5pt \hspace{5pt}}}
% \newshadedtheorem{definition}{Definition}[chapter]

% \def\theoremframecommand{{\color{RoyalBlue!70}\vrule width 5pt \hspace{5pt}}}
% \newshadedtheorem{theorem}{Theorem}[chapter]

% \def\theoremframecommand{{\color{RoyalBlue!30}\vrule width 5pt \hspace{5pt}}}
% \newshadedtheorem{proof}{Proof}[chapter]

% \renewcommand\qedsymbol{$\blacksquare$}           % use black square as QED

% \def\exampleautorefname{Example}

% \def\definitionautorefname{Definition}

% Redefinition of theoremstyle break used for defined environments

\makeatletter
\renewtheoremstyle{break}%
  {\item[\rlap{\vbox{\hbox{\hskip\labelsep \theorem@headerfont
          ##1\ ##2\theorem@separator}\hbox{\strut}}}]}%
  {\item[\rlap{\vbox{\hbox{\hskip\labelsep \theorem@headerfont
%          ##1\ ##2\ (##3)\theorem@separator}\hbox{\strut}}}]}% DELETED
           ##1\ ##2:\ ##3\theorem@separator}\hbox{\strut}}}]}% NEW
\makeatother



\definecolor{defcolor}{rgb}{0.36, 0.54, 0.66}

\definecolor{lemmacolor}{rgb}{0.66, 0.89, 0.86}

\definecolor{corollarycolor}{rgb}{0.13, 0.67, 0.8}

\definecolor{thmcolor}{rgb}{0.29, 0.70, 0.70}

\definecolor{axiomcolor}{rgb}{1, 0.84, 0}
%%%%%%%%%%%%%%%%%%%%%%%%%%%%%%%%%%%%%%%%%%%%%%%%
% A theorem environment
%%%%%%%%%%%%%%%%%%%%%%%%%%%%%%%%%%%%%%%%%%%%%%%%
\theoremheaderfont{\normalfont\bfseries}
\theorembodyfont{\normalfont}
\theoremsymbol{}
\theoremstyle{break}
\def\theoremframecommand{{\color{thmcolor}\vrule width 3pt \hspace{5pt}}}
\newshadedtheorem{theorem}{Theorem}[chapter]
\newenvironment{theo}[1][\unskip]{%
		\begin{theorem}[#1]
}{%
		\end{theorem}
}
\def\theoremautorefname{Theorem}
%%%%%%%%%%%%%%%%%%%%%%%%%%%%%%%%%%%%%%%%%%%%%%%%
% A Proof environment
%%%%%%%%%%%%%%%%%%%%%%%%%%%%%%%%%%%%%%%%%%%%%%%%

\theoremheaderfont{\normalfont\itshape\bfseries}
\theorembodyfont{\normalfont}
\theoremsymbol{\rule{1ex}{1ex}}
\theoremstyle{break}
\newtheorem*{proof}{Proof}[theorem]
\newenvironment{pro}[1][\unskip]{%
		\begin{proof}[#1]
}{%
        
		\end{proof}
}
\def\proofautorefname{Proof}

%%%%%%%%%%%%%%%%%%%%%%%%%%%%%%%%%%%%%%%%%%%%%%%%
% A Definition environment
%%%%%%%%%%%%%%%%%%%%%%%%%%%%%%%%%%%%%%%%%%%%%%%%
\theoremheaderfont{\normalfont\bfseries}
\theorembodyfont{\normalfont}
\theoremsymbol{}
\theoremstyle{break}
\def\theoremframecommand{{\color{defcolor}\vrule width 3pt 
\hspace{5pt}}}
\newshadedtheorem{definition}[theorem]{Definition}
\newenvironment{defi}[1][\unskip]{%
		\begin{defintion}[#1]
}{%
		\end{definition}
}
\def\definitionautorefname{Definition}

%%%%%%%%%%%%%%%%%%%%%%%%%%%%%%%%%%%%%%%%%%%%%%%%
% An axiom environment
%%%%%%%%%%%%%%%%%%%%%%%%%%%%%%%%%%%%%%%%%%%%%%%%
\theoremheaderfont{\normalfont\bfseries}
\theorembodyfont{\normalfont}
\theoremsymbol{}
\theoremstyle{break}
\def\theoremframecommand{{\color{axiomcolor}\vrule width 3pt \hspace{5pt}}}
\newshadedtheorem{axiom}[theorem]{Axiom}
\newenvironment{axi}[1][\unskip]{%
		\begin{axiom}[#1]
}{%
		\end{axiom}
}
\def\axiomautorefname{Axiom}

%%%%%%%%%%%%%%%%%%%%%%%%%%%%%%%%%%%%%%%%%%%%%%%%
% A Lemma environment
%%%%%%%%%%%%%%%%%%%%%%%%%%%%%%%%%%%%%%%%%%%%%%%%
\theoremheaderfont{\normalfont\bfseries}
\theorembodyfont{\normalfont}
\theoremsymbol{}
\theoremstyle{break}
\def\theoremframecommand{{\color{lemmacolor}\vrule width 3pt \hspace{5pt}}}
\newshadedtheorem{lemma}[theorem]{Lemma}
\newenvironment{lemm}[1][\unskip]{%
		\begin{lemma}[#1]
}{%
		\end{lemma}
}
\def\lemmaautorefname{Lemma}

%%%%%%%%%%%%%%%%%%%%%%%%%%%%%%%%%%%%%%%%%%%%%%%%
% A Corollary environment
%%%%%%%%%%%%%%%%%%%%%%%%%%%%%%%%%%%%%%%%%%%%%%%%
\theoremheaderfont{\normalfont\bfseries}
\theorembodyfont{\normalfont}
\theoremsymbol{}
\theoremstyle{break}
\def\theoremframecommand{{\color{corollarycolor}\vrule width 3pt \hspace{5pt}}}
\newshadedtheorem{corollary}[theorem]{Corollary}
\newenvironment{corol}[1][\unskip]{%
		\begin{corollary}[#1]
}{%
		\end{corollary}
}
\def\corollaryautorefname{Corollary}
%%%%%%%%%%%%%%%%%%%%%%%%%%%%%%%%%%%%%%%%%%%%%%%%
% An example environment
%%%%%%%%%%%%%%%%%%%%%%%%%%%%%%%%%%%%%%%%%%%%%%%%
\theoremheaderfont{\normalfont\bfseries}
\theorembodyfont{\normalfont}
\theoremsymbol{}
\theoremstyle{break}
\theoremprework{\bigskip\hrule}
\theorempostwork{\hrule\bigskip}
\newtheorem{example}[theorem]{Example}
\newenvironment{exa}[1][\unskip]{
\begin{example}[#1]
}{
\end{example}
}
\def\exampleautorefname{Example}

% Sets --------------------------------------------------------------------------
\newcommand{\N}{\mathbb{N}}  % natural numbers
\newcommand{\Z}{\mathbb{Z}}  % integers
\newcommand{\Q}{\mathbb{Q}}  % rational numbers
\newcommand{\R}{\mathbb{R}}  % real numbers
\newcommand{\C}{\mathbb{C}}  % complex numbers
\newcommand{\I}{\mathbb{I}}




\let\cite=\citep
\renewcommand{\exp}[1]{e^{#1}}


% Colours
\definecolor{aaublue}{RGB}{33,26,82}
\usepackage[table]{xcolor}



% Headers and footers -----------------------------------------------------------
\fancyhead{}                      % clear default header fields
\setlength{\headheight}{14pt}
\fancyhead[LE,RO]{\projectgroup}  % left on even, right on odd
\fancyhead[RE]{\leftmark}         % chapter name
\fancyhead[LO]{\rightmark}        % sections name
\fancyfoot{}                      % clear default footer fields
\fancyfoot[CE,CO]{\thepage}       % centered page numbers
\pagestyle{fancy}                 % activate fancy style


% Chapter titles ----------------------------------------------------------------
% http://mirrors.dotsrc.org/ctan/macros/latex/contrib/titlesec/titlesec.pdf
\titleformat
{\chapter}
[hang]
{\Huge\bfseries}
{\thechapter\enspace{|}\enspace}
{0pt}
{\Huge\bfseries}




\numberwithin{equation}{chapter}  % resets the counter of equation number each time a new \section is used

\usepackage{subcaption}
\usepackage{caption}
\usepackage{listings}
\usepackage{comment}
\usepackage{advdate}
\usepackage{dcolumn}
\usepackage{amsfonts}

\captionsetup[figure]{font={it}}
\captionsetup[table]{font={it}}
% table
\renewcommand{\arraystretch}{1.5}

%include bibliography in the table of content
\usepackage[nottoc]{tocbibind}


% Matrix v_line
\makeatletter
\renewcommand*\env@matrix[1][*\c@MaxMatrixCols c]{%
   \hskip -\arraycolsep
   \let\@ifnextchar\new@ifnextchar
   \array{#1}}
\makeatother
% Arrows for text 
%\usetikzlibrary{positioning, arrows.meta}
\tikzset{%
    myarrow/.style = {-Stealth, shorten >=5pt}
}
\usetikzlibrary{shapes,shadows,arrows,positioning,graphs,shapes.geometric,arrows.meta}
\usetikzlibrary{calc,intersections,through,hobby}
\usetikzlibrary{matrix}
\tikzset{point/.style={circle,inner sep=0pt,minimum size=3pt,fill=red}}
\newcommand{\mypoint}[2]{\tikz[remember picture]{\node[inner sep=0, anchor=base](#1){$#2$};}}
%Fancy circles.
\newcommand*\circled[1]{\tikz[baseline=(char.base)]{
            \node[shape=circle,draw,inner sep=2pt] (char) {#1};}}
%pagebreak align
\allowdisplaybreaks

%For pseudo code
\usepackage{algorithm,algpseudocode,float}
\usepackage{lipsum}
\usepackage[noend]{algpseudocode}
\makeatletter
\newenvironment{breakablealgorithm}
  {% \begin{breakablealgorithm}
   \begin{center}
     \refstepcounter{algorithm}% New algorithm
     \hrule height.8pt depth0pt \kern2pt% \@fs@pre for \@fs@ruled
     \renewcommand{\caption}[2][\relax]{% Make a new \caption
       {\raggedright\textbf{\ALG@name~\thealgorithm} ##2\par}%
       \ifx\relax##1\relax % #1 is \relax
         \addcontentsline{loa}{algorithm}{\protect\numberline{\thealgorithm}##2}%
       \else % #1 is not \relax
         \addcontentsline{loa}{algorithm}{\protect\numberline{\thealgorithm}##1}%
       \fi
       \kern2pt\hrule\kern2pt
     }
  }{% \end{breakablealgorithm}
     \kern2pt\hrule\relax% \@fs@post for \@fs@ruled
   \end{center}
  }
\makeatother

\makeatletter
\def\BState{\State\hskip-\ALG@thistlm}
\makeatother

\usepackage{cleveref}
\usepackage[american]{circuitikz}

\renewcommand{\b}[1]{%
  \ensuremath{\left[#1\right]}%
}

\newcommand{\p}[1]{%
  \ensuremath{\left(#1\right)}%
}

\newcommand{\diag}[1]{%
  \ensuremath{\text{diag}\left(#1\right)}%
}

\newcommand{\floor}[1]{%
  \ensuremath{\left\lfloor #1 \right\rfloor}%
}

\newcommand{\sigm}{%
  \ensuremath{\Sigma^{-1}}%
}

\newcommand{\sigtilde}{%
  \ensuremath{\Tilde{\Sigma}}%
}
\renewcommand{\d}{%
  \ensuremath{\underline{d}}%
}
\renewcommand{\mathbf}[1]{%
  \ensuremath{\boldsymbol{#1}}%
}
\newcommand{\D}{%
  \ensuremath{\underline{D}}%
}
\newcommand{\x}{%
  \ensuremath{\underline{x}}%
}


\newcommand{\ls}{%
  \ensuremath{\limsup_{n\rightarrow \infty}}%
}

\renewcommand{\part}[2]{%
  \ensuremath{\frac{\partial #1}{\partial #2}}%
}

\newcommand{\mpart}[2]{%
  \frac{\partial #1}{\partial #2}%
}




\usepackage{stackengine}
\def\defeq{\mathrel{\ensurestackMath{\stackon[2pt]{=}{\scriptstyle\Delta}}}}

\newcommand\scalemath[2]{\scalebox{#1}{\mbox{\ensuremath{\displaystyle #2}}}}

\usepackage{tikz-3dplot}